\section{Odvození transformačního klíče vybraných 2D transformací}

Cílem této části je podrobně popsat všechny implementované postupy výpočtu
transformačních koeficientů. Popsán bude jak obecný postup vyrovnání, tak
konkrétní tvary matic pro jednotlivé transformační metody.
\subsection{Metody vyrovnání}
\label{kapvyr}
Pro jednoznačné určení transformačního klíče z nadbytečného počtu identických bodů byla použita metoda
nejmenších čtverců. Tato metoda byla aplikovaná na vzdálenost (v rovinně) mezi
cílovými a přetransformovanými výchozími body. 

V následujícím textu budou popsány vzorce, které byly použité pro jednotlivá
vyrovnání metodou zprostředkujících pozorovaní. Dále bude také zmíněn
složitější případ vyrovnaní metodu zprostředkujících s podmínkami.

Pro další rovnice použijeme toto označení:

\begin{table}[h]
\centering

\begin{tabular}{llll}
$ {\bf x} $&$=$& $(x_1 \ldots x_k)^T$  & vektor hledaných veličin\\
$ {\bf x_0} $&$=$&$(x_{10} \ldots x_{k0})^T$  & vektor přibližných hodnot
neznámých\\
$ d {\bf x} $&$=$&$(dx_1 \ldots dx_k)^T$  & vektor přírůstků přibližných
hodnot\\

$ F_i(\bf x)$ &&& funkční vztah mezi hledanými $ \bf x$ a měřenými
veličinami \\
&&&(v našem případě transformační rovnice)\\
$ l_i$ &&& naměřená zprostředkující veličina \\
$ v_i$ &&& oprava naměřené zprostředkující veličina \\
$ \bar{f}_i$ &&& vyrovná hodnota zprostředkující veličiny \\
$i$ &&& index $i$-té zprostředkující rovnice
\end{tabular}
\end{table}

Pak platí:

$$F_i ({\bf x}) = \bar{f}_i$$
$$F_i (x_1 \ldots x_k) = l_i + v_i$$
$$F_i (x_{10} +dx_1 \ldots x_{k0} +dx_k) = l_i + v_i$$

Levou stranu rozvedeme do Taylerova rozvoje \footnote[1]{
Pro případ, kdy zavedeme vhodnou substituci a např. jako transformační
koeficienty nevolíme přímo geometrické parametry, není použití Taylerova 
rozvoje nutné. V případě, kdy potřebujeme, ale zavést podmínky přímo pro
geometrické koeficienty (např, stočení rovno nule,  nebo shodná změna měřítek
pro obě osy apod..) Tylerovo rozvoj musíme použí.}:

$$F_i ({\bf x}_0) +  \left(\frac{F_i}{dx_1}\right)dx_1 +\ldots+\left(\frac{F_i
}{dx_k}\right)dx_k = l_i + v_i$$

Po úpravě dostáváme:
\begin{eqnarray} \label{aaa}
 a_{i1} dx_1 + a_{i2} dx_2 + \ldots + a_{ik} dx_k + F_i ({\bf x_0})-l_i = v_i
\end{eqnarray}

kde $i = 1, \ldots ,n$ a  $n$ je počet linearizovaných zprostředkujících rovnic
oprav a $k$ je počet hledaných neznámých. Výraz $ F_i ( {\bf x_0} )-l_i= L_i$ je
tzv. absolutní člen.

Výraz \ref{aaa} můžeme zapsat maticově takto:

$${\bf A} d {\bf x} + {\bf L} = {\bf v}, $$

kde

\begin{eqnarray} {\bf A} = \left(
\begin{array}{cccc }
a_{11}  &   a_{21}  & \ldots  & a_{1k}  \\
a_{21}  &   a_{22}  & \ldots  &a_{2k} \\
\vdots  &   \vdots  & \ddots  & \vdots \\
a_{nk} &    a_{n2}  & \ldots  &a_{nk}                                
\end{array}
 \right), 
\hspace{0.5cm}
d \bf x =
\left(
\begin{array}{c }
 dx_1 \\
 dx_2\\
 \vdots\\
 dx_k                        
\end{array}
 \right), 
\end{eqnarray} 

\begin{eqnarray} 
 \label{L}
\bf L =
\left(
\begin{array}{c }
 F_1 (\bf x_0) - l_1 \\
 F_2 (\bf x_0) - l_2 \\
 \vdots\\
 F_n (\bf x_0) - l_n \\                   
\end{array}
 \right).
\end{eqnarray}

Po aplikace metody nejmenších čtverců (odvození viz \cite{vyrovnani}) získáváme
pro vyrovnané neznámé vztah:

\begin{eqnarray} \label{ata}
 d{\bf x} = -({\bf A}^T {\bf A} )^{-1}({\bf A}^T {\bf L}) 
\end{eqnarray}

Tento způsob byl implementován pro výpočet podobností, afinní a projektivní
transformace, přičemž konkrétní matice jsou uvedeny v následujících kapitolách.
Po případ, kdy chceme zavést podmínky pro neznáme je výpočet následující:

Podmínky pro opravy neznámých definujeme pomocí obecného vyjádření:

\begin{eqnarray} \label{podminky}
b_{11} v_1 + \ldots &+ b_{1n} v_n + U_1=0 \\\nonumber
& \vdots &\\
b_{r1} v_1 + \ldots &+ b_{rn} v_n + U_r=0 \nonumber
\end{eqnarray}

kde U jsou hodnoty uzávěrů - tzn. rozdíl mezi hodnotou vypočítanou z
přibližných hodnot a hodnotou požadovanou, která je stanovena podmínkou.

Maticově:

\begin{eqnarray} \label{podminky} 
{\bf Bv} + \bf{U} = {\bf 0}
\end{eqnarray}

Pro společné vyrovnání zprostředkujících měření s podmínkami pak dostáváme
(odvození viz \cite{vyrovnani}):

\begin{eqnarray} \label{vyropr}
\left(
\begin{array}{c}
d{\bf x}\\
\bf k\\ 
\end{array} \right) 
=  \left( 
\begin{array}{cc}
\bf A ^T \bf A & \bf B^T \\
\bf B          & \bf 0
\end{array}\right)^{-1}
\left( 
\begin{array}{c}
\bf A ^T \bf L \\
\bf U
\end{array}\right) 
\end{eqnarray}

